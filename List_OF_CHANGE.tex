% List of changes for QIRT_JOURNAL 2016
\documentclass{article}

\usepackage{fontspec}

\begin{document}
    
\title{List of changes for QIRT Journal No. TQRT-2017-0004}
\author{Lei Lei}

\maketitle


\section{Answers to reviewers} % (fold)
\label{sec:answers_to_reviewers}
\subsection{Reviewer 1}

\begin{quote}
    ``On my opinion the paper is not plausible for publishing in QIRT Journal because the scientific niveau is not adequate and the basic approach given by eq. (4) is mathematicaly incorrect."
\end{quote}
\textbf{Answer}: \\
The Eq. (4) is mathematically incorrect because in the submitted version Page 3, Line 47:
``where $\Delta \theta_{x,y}=(\theta_{wi} (x,y)- \theta_{i}) $ and $ \Delta \theta_r =\theta_{wi}(x_r,y_r)- \theta_i $."\\
An error is found in the defintion of $\Delta \theta_{x,y}$. 
It should be $\Delta \theta_{x,y}=[\theta_{wi} (x,y)- \theta_{wi} (x_r,y_r)] $ and it is corrected in the revised paper. The Eq.(3) and Eq.(4) have been simplified for avoiding confusions.

\subsection{Reviewer 2}
\begin{quote}
    ``* There is very small difference between two images in Fig. 4. Is it really necessary to introduce spherical projection? This point should be better explained."
\end{quote}
\textbf{Answer}: \\
The small difference in Fig. 4 is because the focus length of the IR camera is $10 mm$, which leads to a less distorted projection of the image.

Because ``Normal panoramas involve complex image-processing including translation, rotation and perspective projection, etc. However, cylindrical and spherical panoramas are commonly used because of their ease of construction. In our case, a spherical panorama is more convenient.”
This part is added in the Sec 4 of the revised paper.


\begin{quote}
    ``* In the 1st line of the Section 6, the number of the second table is omitted."
\end{quote}
\textbf{Answer}: \\
Corrected in the revised paper. Please see the list of changes. 

\begin{quote}
    ``* Page 8. The 3rd line from the bottom seems to be not finished."
\end{quote}
\textbf{Answer}: \\
Corrected in the revised paper. Please see the list of changes. 

\begin{quote}
    ``* Page 10. In the formula for e the results written as 0.0087 = 0.87 contains an error. Should be 0.87\%."
\end{quote}
\textbf{Answer}: \\
Corrected in the revised paper. Please see the list of changes. 

\begin{quote}
    ``* Even if the paper is clearly written, some English editing is recommended (broken and/or unfinished sentences)."
\end{quote}
\textbf{Answer}: \\
Polished in the revised paper. 

\begin{quote}
    ``* Reference [12]. What is ????"
\end{quote}
\textbf{Answer}: \\
Corrected in the revised paper (which is the year of the reference). Please see the list of changes. 
% section answers_to_reviewers (end)




\section{List of changes in the revised paper} % (fold)
The entire article has been polished. Main changes can be found in the following:
% section section_name (end)
\begin{enumerate}
    \item Page 1, the adresses of two affillation are added; 
    \item Page 3, Eq.(3) and Eq.(4) are rewritten for simplification. The heat flux $q_r$ direction is changed properly for the reason that it propagated from inside to outside.
    \item Page 3, line 47, the definition of $\Delta \theta_{x,y}$ is corrected as $\Delta \theta_{x,y}=[\theta_{wi} (x,y)- \theta_{wi} (x_r,y_r)] $; 
    \item Page 5, from line 7, added ``Normal panoramas involve complex image-processing including translation, rotation and perspective projection, etc. However, cylindrical and spherical panoramas are commonly used because of their ease of construction. In our case, a spherical panorama is more convenient.”;
    \item Page 8, Figure number is added in the 3rd line from the bottom;
    \item Page 10, added \% in the result of the formula of $e$;
    \item Page 11, added the years of ref [12] and [13].
    % \item Modify the title of Figure 7;
    % \item Add the sense of "n" in Equation 2;
    % \item Add subfigure titles from Figure 3 to Figure 9 to make them more logical;
    % \item Add project details in Acknowledgment.
\end{enumerate}

\end{document}
